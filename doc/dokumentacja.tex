\documentclass{article}
\usepackage{verbatim}
\usepackage[MeX]{polski}
\usepackage[utf8]{inputenc}
\usepackage{mathtools}
\author{Piotr Izert, Łukasz Dragan}
\title{Algorytmy Zaawansowane - POLE}

\begin{document}

\maketitle
\newpage
\tableofcontents
\newpage
\section{Przedstawienie problemu}

\subsection{Treść zadania}
\paragraph{}
\textit{
Zaprojektować i zaimplementować algorytm, który w czasie liniowym względem n oblicza pole n-wierzchołkowego prostego wielokąta oraz sprawdza, czy podany punkt leży wewnątrz tego wielokąta. Program powinien zawierać procedurę sprawdzającą, czy dany wielokąt jest prosty.
}
\subsection{}

\section{Opis rozwiązania}
\section{Analiza poprawności}
\section{Opis wejścia/wyjścia}
\subsection{Wejście}
\paragraph{}Program domyślnie jako wejście przyjmuje zawartość pliku ,,in.txt'', który powinien zawierać w kolejnych liniach:
\begin{enumerate}
\item Dane postaci \(x_1\ y_1\ ...\ x_n\ y_n\), gdzie \((x_i,y_i) \in \Re^{2} \ dla\ i=1,2,...,n\) to współrzędne kolejnych punktów a \(n\) to liczba wierzchołków wielokąta.
\item Dane postaci \(x\ y\), gdzie \((x,y) \in \Re^{2}\) będące współrzędnymi punktu, którego zawieranie w wielokącie na zostać sprawdzone.
\end{enumerate}
\paragraph{Przykładowe wejście} \mbox{}\\
\texttt{
344,8 91,2 68,8 121,6 352,8 218,4 448 114,4\\
288,8 136
}

\subsection{Wyjście}
\paragraph{}Rezultat działania programu zapisywany jest w pliku ,,out.txt'' w postaci \(S\ Ans\) gdzie \(S\) to pole powierzchni wielokąta a \(Ans\in\{"TAK","NIE"\}\) to odpowiedź na pytanie, czy dany punkt jest zawarty w wielokącie. W przypadku, gdy dany wielokąt nie jest prosty rezultatem działania programu jest \texttt{NOT SIMPLE}. Jeżeli dane podane na wejściu są niepoprawne, program zapisze do pliku \texttt{BAD INPUT}.

\paragraph{Przykładowe wyjście} \mbox{}\\
\texttt{1243,33 TAK}

\end{document}
