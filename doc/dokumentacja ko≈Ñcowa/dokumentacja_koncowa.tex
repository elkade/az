\documentclass{article}
\usepackage{verbatim}
\usepackage[MeX]{polski}
\usepackage[utf8]{inputenc}
\usepackage{mathtools}

\usepackage{float}

\newcommand{\tab}[1]{\hspace{.05\textwidth}\rlap{#1}}

\author{Piotr Izert, Łukasz Dragan}
\title{Algorytmy Zaawansowane - POLE\\Dokumentacja końcowa}

\begin{document}

\maketitle

\newpage
\tableofcontents
\newpage

\section{Opis zmian}
Nie zostały wprowadzone żadne zmiany względem pierwotnych założeń działania algorytmów i programu.
\section{Instrukcja użytkownika}
Dane wejściowe programu mogą zostać wprowadzone poprzez plik tekstowy lub za pomocą interfejsu graficznego.\\
Interfejs graficzny składa się z części przeznaczonej do rysowania oraz trzech przycisków zwykłych i dwóch przycisków typu ,,radio''.
\subsection{Rysowanie}
Ekran rysowania ma wymiary 500x500 a środek układu współrzędnych znajduje się w lewym górnym rogu. W celu narysowania wielokąta należy wybrać opcję ,,polygon'' i klikając na ekranie rysowania tworzyć kolejne punkty wielokąta. Podwójne wciśnięcie przycisku myszy kończy rysowanie. W celu narysowania nowego wielokąta należy wykonać opisaną operację ponownie, a stary wielokąt zniknie samoczynnie.\\
W celu narysowania punktu należy przy wybranej opcji ,,point'' kliknąć na ekran rysowania. Zostanie wybrany punkt. Aby wybrać nowy punkt należy kliknąć w inne miejsce ekranu. Wówczas stary punkt zniknie. Interfejs zawiera również przycisk ,,Clear'' służący do czyszczenia ekranu rysowania i pamięci operacyjnej programu.
\subsection{Operacje na plikach}
Po wybraniu przycisku ,,read input'' program sprawdza, czy w jego lokalizacji istnieje plik ,,in.txt''. Jeżeli tak, to wczytuje z niego dane i prezentuje w formie graficznej. W przeciwnym wypadku ekran rysowania jest pusty.
\subsection{Obliczanie wyniku}
Przy wprowadzonych danych widocznych w części graficznej możliwe jest dokonanie obliczeń: pola wielokąta, sprawdzenia zawierania punktu w wielokącie oraz sprawdzenia, czy wielokąt jest prosty. W tym celu należy wybrać przycisk ,,Calculate''. Wyświetlone zostanie wówczas okno z wartościami obliczonych danych oraz dane te zostaną zapisane do pliku ,,out.txt''.
\section{Opis testów}



\section{Podział prac}
\subsection{Łukasz Dragan}
\begin{itemize}
\item Implementacja interfejsu graficznego i obsługi plików,
\item Implementacja i dokumentacja algorytmu liczenia pola wielokąta.
\end{itemize}

\subsection{Piotr Izert}
\begin{itemize}
\item Implementacja i dokumentacja algorytmu sprawdzania, czy wielokąt jest prosty,
\item Implementacja i dokumentacja algorytmu sprawdzania, czy punkt znajduje się wewnątrz wielokąta.
\end{itemize}









\section{Opis wejścia/wyjścia}

\subsection{Wejście}

\paragraph{}
Program domyślnie jako wejście przyjmuje zawartość pliku ,,in.txt'', który powinien zawierać w kolejnych liniach:
\begin{enumerate}
\item Dane postaci \(x_1\ y_1\ ...\ x_n\ y_n\), gdzie \((x_i,y_i) \in \Re^{2} \ dla\ i=1,2,...,n\) to współrzędne kolejnych punktów a \(n\) to liczba wierzchołków wielokąta.
\item Dane postaci \(x\ y\), gdzie \((x,y) \in \Re^{2}\) będące współrzędnymi punktu, którego zawieranie w wielokącie ma zostać sprawdzone.
\end{enumerate}

\paragraph{Przykładowe wejście} \mbox{}\\
\texttt{
344,8 91,2 68,8 121,6 352,8 218,4 448 114,4\\
288,8 136
}

\subsection{Wyjście}

\paragraph{}
Rezultat działania programu zapisywany jest w pliku ,,out.txt'' w postaci \(S\ Ans\) gdzie \(S\) to pole powierzchni wielokąta a \(Ans\in\{"TAK","NIE"\}\) to odpowiedź na pytanie, czy dany punkt jest zawarty w wielokącie. W przypadku, gdy dany wielokąt nie jest prosty, rezultatem działania programu jest \texttt{NOT SIMPLE}. Jeżeli dane podane na wejściu są niepoprawne, program zapisze do pliku \texttt{BAD INPUT}.

\paragraph{Przykładowe wyjście} \mbox{}\\
\texttt{1243,33 TAK}

\end{document}
